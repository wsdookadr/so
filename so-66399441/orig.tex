\documentclass{article}
\usepackage{pythontex}
\usepackage{mathtools,amssymb}
\usepackage{amsmath}
\usepackage{enumitem}

\begin{document}

\begin{pycode}
import math
from sympy import *
from random import randint, seed

seed(2021)
\end{pycode}

\paragraph{Oppgave 3}

\begin{pycode}
a, b = randint(1,2), 3
ab = Rational(a,b)

pressure_num = lambda x: 1-x
pressure_denom = lambda x: 1+x

def pressure(x):
  return (1-x)/(1+x)

pressure_ab = Rational(pressure_num(ab),pressure_denom(ab))

x, y, z = symbols('x y z')
pressure_derivative = simplify(diff(pressure(x), x))
pressure_derivative_ab = pressure_derivative.xreplace({ x : Rational(a,b)}) 
\end{pycode}

The partial pressure of some reaction is given as
%
\begin{pycode}
print(r"\begin{align*}")
print(r"\rho(\zeta)")
print(r"=")
print(latex(pressure(Symbol('\zeta'))))
print(r"\qquad \text{for} \ 0 \leq \zeta \leq 1.")
print(r"\end{align*}")
\end{pycode}
%
\begin{enumerate}[label=\alph*)]
    \item Evaluate $\rho(\py{a}/\py{b})$. Give a physical interpretation of your
        answer.
    \begin{equation*}
        \rho(\py{a}/\py{b})
        = \frac{1-(\py{ab})}{1+\py{ab}}
        = \frac{\py{pressure_num(ab)}}{\py{pressure_denom(ab)}}
        \cdot \frac{\py{b}}{\py{b}}
        = \py{pressure_ab}
    \end{equation*}
\end{enumerate}

The derivative is given as
%
\begin{pycode}
print(r"\begin{align*}")
print(r"\rho'({})".format(ab))
print(r"=")
print(latex(pressure_derivative))
print(r"=")
print(latex(simplify(pressure_derivative_ab)))
print(r"\end{align*}")
\end{pycode}

\end{document}
